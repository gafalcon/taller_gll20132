% !TEX TS-program = pdflatex
% !TEX encoding = UTF-8 Unicode

% This is a simple template for a LaTeX document using the "article" class.
% See "book", "report", "letter" for other types of document.

\documentclass[11pt]{article} % use larger type; default would be 10pt

\usepackage[utf8]{inputenc} % set input encoding (not needed with XeLaTeX)

%%% Examples of Article customizations
% These packages are optional, depending whether you want the features they provide.
% See the LaTeX Companion or other references for full information.

%%% PAGE DIMENSIONS
\usepackage{geometry} % to change the page dimensions
\geometry{a4paper} % or letterpaper (US) or a5paper or....
% \geometry{margin=2in} % for example, change the margins to 2 inches all round
% \geometry{landscape} % set up the page for landscape
%   read geometry.pdf for detailed page layout information

\usepackage{graphicx} % support the \includegraphics command and options

% \usepackage[parfill]{parskip} % Activate to begin paragraphs with an empty line rather than an indent

%%% PACKAGES
\usepackage{booktabs} % for much better looking tables
\usepackage{array} % for better arrays (eg matrices) in maths
%\usepackage{paralist} % very flexible & customisable lists (eg. enumerate/itemize, etc.)
\usepackage{verbatim} % adds environment for commenting out blocks of text & for better verbatim
\usepackage{subfig} % make it possible to include more than one captioned figure/table in a single float
% These packages are all incorporated in the memoir class to one degree or another...

%%% HEADERS & FOOTERS
\usepackage{fancyhdr} % This should be set AFTER setting up the page geometry
\pagestyle{fancy} % options: empty , plain , fancy
\renewcommand{\headrulewidth}{0pt} % customise the layout...
\lhead{}\chead{}\rhead{}
\lfoot{}\cfoot{\thepage}\rfoot{}

%%% SECTION TITLE APPEARANCE
\usepackage{sectsty}
\allsectionsfont{\sffamily\mdseries\upshape} % (See the fntguide.pdf for font help)
% (This matches ConTeXt defaults)

%%% ToC (table of contents) APPEARANCE
\usepackage[nottoc,notlof,notlot]{tocbibind} % Put the bibliography in the ToC
\usepackage[titles,subfigure]{tocloft} % Alter the style of the Table of Contents
\renewcommand{\cftsecfont}{\rmfamily\mdseries\upshape}
\renewcommand{\cftsecpagefont}{\rmfamily\mdseries\upshape} % No bold!

%%% END Article customizations

\usepackage[spanish]{babel}
\usepackage{listings} 
\usepackage{color}
%%% The "real" document content comes below...


\title{\color{red}Investigación de Lenguajes - ML}
\author{Gabriel Falcones\and Angel Espin}
%\date{} % Activate to display 1a given date or no date (if empty),
% otherwise the current date is printed 
\lstset{language=ML}
\begin{document}
\maketitle
%\tableofcontents % No hace falta un TOC en un artículo corto


\section{\color{red}Introducción}
ML es un lenguaje de programación de tipo funcional de tipificado estático, desarrollado en 1973 por Robert Milner en la Universidad de Edimburgo. Su nombre es un acrónimo de Meta Lenguaje, desarrollado para realizar demostraciones automáticas de teoremas en el sistema LCF.\\

Difiere de otros lenguajes de programación funcional como Haskell, en que también permite la programación de tipo imperativa, al poseer identificadores, que tienen la apariencia de nombres de variables en lenguages imperativos.\\

ML ha producido varios dialectos usados en la actualidad, entre ellos destacan Standard ML y OCaml. Además ha sido influencia para otros lenguajes como Haskell,Miranda, Cyclone, Clojure e incluso C++.\\

Las fortalezas de ML son en su mayoría aplicadas en la manipulación y diseño de lenguajes(compiladores y analizadores), pero es un lenguaje de propósito general aplicado también en sistemas financieros, programacion cliente/servidor, bioinformática, etc
  

\section{\color{red}Características}
ML, al ser un lenguage funcional, fue modelado de manera similar a las funciones matemáticas. En vez de usar variables y asignaciones, se enfoca en la aplicación de funciones, condicionales y de la recursión para el control de la ejecución, y formas funcionales para construir funciones complejas.\\
 
Utiliza una sintaxis más relacionada a los lenguajes imperativos que a otros lenguajes funcionales como LISP.
ML incluye análisis estático de tipos, inferencia de tipos,polimorfismo parametrizado, manejo de excepciones, una variedad de estructuras de datos, y tipos de datos abstractos.\\

ML posee también declaración de tipos, pero debido a la característica de inferencia de tipos, por lo general no se lo utiliza.\\Considere la siguiente función:\\
\begin{lstlisting}[frame=single]
fun square(x) = x * x;
\end{lstlisting}
\\
ML determina el tipo de dato tanto del parámetro como del valor que retorna la función gracias al operador *. Debido a que es un operador aritmético, se asume que tanto el parámetro x y el valor que retorna la función son numéricos, por defecto asume que son tipo int. 
Ml no permite la coercion de parámetros: no es posible siquiera usar un entero cuando se espera un float, sin hacer el casting explicitamente. Tampoco permite sobrecarga de funciones.\\
Si la función anterior fuese usada de la siguiente manera:\\
\begin{lstlisting}[frame=single]
square(2.75);
\end{lstlisting}
\\
Esto causaría un error, porque ML no coerciona valores reales a valores enteros. Si quisieramos que la función square acepte valores reales, deberíamos escribirla de la siguiente manera:\\
\begin{lstlisting}[frame=single]
fun square(x:real) = x * x;
\end{lstlisting}\\

Y debido a que ML no soporta la sobrecarga de funciones, esta versión no podría coexistir con la versión int anterior.\\

Es posible usar el llamado de patrones (pattern matching) para realizar multiples definiciones de funciones. Las diferentes definiciones de una función que dependen de la forma del parámetro son separadas por un símbolo {\bf OR} ( \textbar ). Por ejemplo, usando pattern matching, la función factorial puede ser escrita de la siguiente manera: \\
\begin{lstlisting}[frame = single]
fun fact(0) = 1
| fact(1) = 1
| fact(n:int) = n * fact(n - 1);
\end{lstlisting}\\

Si {\bf fact} es llamada con el parámetro 0, la primera definición es usada, si el parámetro es 1, utiliza la segunda definición, para cualquier otro caso, utilizará la tercera definición.

ML soporta Currificación(Currying), que es el proceso de reemplazar una función que acepta múltiples parámetros con otra función que acepta solo un parámetro y retorna otra función que toma los otros parámetros.\\
Por ejemplo:\\
\begin{lstlisting}[frame=single]
fun add a  = fn b => a + b;
\end{lstlisting}\\

Esta función toma un parámetro entero (a), y devuelve una función que también toma un parámetro entero (b). Una llamada a esta función no necesita las comas (,) para separar los parámetros, de la siguiente manera:\\

\begin{lstlisting}[frame=single]
add 4 5  (* El llamado a esta funcion devuelve 8 *) 
\end{lstlisting}

Esta misma función {\bf add} puede escribirse de una manera más sencilla:\\

\begin{lstlisting}[frame=single]
(* Syntactic sugar para una funcion que utiliza Currying *)
fun add a b = a + b;
\end{lstlisting}\\

A diferencia de Haskell que utiliza lazy evaluation, ML utiliza eager evaluation(o greedy evaluation), lo que significa que todas las subexpresiones son siempre evaluadas.

\section{\color{red}Historia}
ML se refiere como lenguaje funcional impuro, ya que este permite los efectos secundarios,  y a la programcaion imperativa
El lenguaje de programacion ML, conocido como SML, inspirado en algunos conceptos fundamentales de la informática; pone a disposición:
\begin{itemize}
\item Los árboles y otros tipos de datos recursivos declarado y pueden utilizarse sin mención de los punteros;
\item Las funciones son valores de ML estándar, las funciones pueden tener funciones como argumentos y devuelven como resultados;
\item Tipo de polimorfismo ( Milner, 1978 ) hace posible para declarar funciones cuyo tipo depende del contexto. Por ejemplo, la misma función puede ser usada para invertir una lista de enteros y una lista de cadenas.
\end{itemize}
En la actualidad es un lenguaje de propósito general
\section{\color{red}Tutorial de Instalación}

\begin{itemize}
\item Vaya al directorio C : \ y descomprimir la distribución ML Moscú
mediante la ejecución

unzip win32 mos20bin.zip

NOTA : Debe utilizar una versión de descompresión hizo preserva archivo largo
nombres, : como Info- Zip o Winzip .

Esto crea un directorio C: \ mOsml con subdirectorios

mOsml / readme , install.txt
bin / mOsml , mosmlc , mosmllex , mosmlyac , camlrunm ,
y las bibliotecas cargadas dinámicamente
copyrght / avisos de copyright
doc / manual.pdf , mosmlref.pdf , mosmllib.pdf , ...
documentación mosmllib / HTML de la biblioteca ML Moscú
examples / algunos programas de ejemplo
lib / bytecode y base de las unidades de la biblioteca
Herramientas / mosmldep , Makefile.stub

\item Añadir C: \ mOsml \ bin a la variable PATH, y establecer el entorno
     variable ` mosmllib ' a C: \ mOsml \ lib .

\begin{itemize}
\item En el caso de Windows 95/98/ME Si su archivo AUTOEXEC.BAT contención algo como:

SET PATH = c: \ dos ; ... C: \ mOsml \ bin
establecer MOSMLLIB = C: \ mOsml \ lib

El sistema se debe reiniciar para que el nuevo entorno variable.

\item En el caso de Windows NT/2000/XP , seleccione Inicio | Configuración | Panel de control | Sistema | Avanzada | Variables de entorno Variables de sistema | | NuevoIntroduzca MOSMLLIB como nombre de variable y C: \ mOsml \ lib como valor de la variable .Haga doble clic en Path y añada C: \ mOsml \ bin a la variable valor .

\end{itemize}

\item Inicio Moscú ML mediante la introducción de la línea de comandos

        mOsml

\end{itemize}

\section{\color{red}Ejemplo: Funcion Factorial}

ML
fun fac (n) =
if n = 0 then 1
else n * fac (n - 1)


\lstset{language=ML}          % Set your language (you can change the language for each code-block optionally)

\begin{lstlisting}[frame=single]  % Start your code-block
fun test x =
case x of
[] => 0
x::xs' => x + test xs'
\end{lstlisting}



\end{document}
